\documentclass[12pt]{article}
\usepackage[T1]{fontenc}

%
%Margin - 1 inch on all sides
%
\usepackage[letterpaper]{geometry}
\geometry{top=1.0in, bottom=1.0in, left=1.0in, right=1.0in}



%
%Doublespacing
%
\usepackage{setspace}
\doublespacing
% 
%Babel package for multiple language typesetting
%
%\usepackage[english,german]{babel}
%\usepackage[T1]{fontenc}
%\usepackage[latin1]{inputenc}
%
%Setting the font
%
\usepackage{times}
%
%Rotating tables (e.g. sideways when too long)
%
\usepackage{rotating}
%
%For multiple rows in tables
%
\usepackage{multirow}
% 
%Line numbering in verse environment
%
\usepackage{lineno} 

%
%Fancy-header package to modify header/page numbering (insert last name)
%
\usepackage{fancyhdr}
\pagestyle{fancy}
\lhead{} 
\chead{} 
\rhead{Turkeltaub \thepage} 
\lfoot{} 
\cfoot{} 
\rfoot{} 
\renewcommand{\headrulewidth}{0pt} 
\renewcommand{\footrulewidth}{0pt} 
%To make sure we actually have header 0.5in away from top edge
%12pt is one-sixth of an inch. Subtract this from 0.5in to get headsep value
\setlength\headsep{0.333in}
%
%Works cited environment
%(to start, use \begin{workscited...}, each entry preceded by \bibent)
% - from Ryan Alcock's MLA style file
%
\newcommand{\bibent}{\noindent \hangindent 40pt}
\newenvironment{workscited}{\newpage \begin{center} Works Cited \end{center}}{\newpage }

%
%Begin document
%
\begin{document}
\begin{flushleft}
%%%%First page name, class, etc
Ethan Turkeltaub\\
M. Renganeschi \\
Writing for College Honors \\
\today\\

%%%%Title
\begin{center}
\textbf{Religion's Exit from Society}
\end{center}

%%%%Changes paragraph indentation to 0.5in
\setlength{\parindent}{0.5in} 
%%%%Begin body of paper here

For thousands of years, religions existed without challenge --- they were a source of hope, of truth, and of support for the people of the world. The often uneducated masses blindly followed whatever their religious leader said to them. However, with the recent advances in science and in moral support, people don't need to rely on their religion as a source of truth. Needless to say, religious leaders who have had this power for many years are reluctant to release their stranglehold on the sheep that follow them.

This does not only apply to the Catholic Church, one of the worst offenders of unchecked power --- it applies to Islam, to Judaism, to Hinduism, to every other religion. In fact, it needn't be a "traditional" religion at all. People have obsessions that could be considered religious in scale: for example, those who are part of the "church" of Scientology are obsessive with L. Ron Hubbard, its founder. Scientology is one of many new religions that does not stem from the Bible. They are not nearly as established as Christianity, Judaism, Islam, Hinduism, etc., but they have a large, very loyal following.

Despite all of these religion's differences, at their core, they have similar organizations. For example, a good majority have a central leader that commands the entire organization's thinking on an issue. They interpret whatever beliefs the religion is based around for the followers. This, plus a loyal following is what I believe defines a religion --- without either of these two parts, there is nothing. Without the central organization, it is simply mob-mentality, sheep following each other without knowing why. Without the following, they are simply labeled as "crazy" --- if a person said there was a supernatural, all-encompassing being that you were talking to and receiving orders from, people would question their mental state. However, once people start following and believing what that person said, it would become a religion.

Society's transition from that of religion to that of science has not been as sudden as some might think. There have been dissidents for thousands of years that challenged the current established religion's claims. Mostly, both sides claims have been without empirical evidence and simply based in a faith or belief. Now, most anti-religious arguments are based solely in science --- atheists and agnostics have empirical evidence to take any religious arguments and immediately disprove it. As astrophysicist and director of the Hayden Planetarium at the Rose Center for Earth and Space says, "The good thing about science is that it's true whether or not you believe in it". 

In our age of postmodernism, religion has no place. If modernism was a departure from traditional ideas and values, then postmodernism takes this and goes a step further with a general mistrust of these traditional ideas and values.

In this sense, the step from religion in the modern era was to create these new religions like Scientology. Now, the step from religion in the postmodern era is foregoing religion entirely. We can see this happening in both statistics (the Pew Research Center has concluded that "nones" are on the rise), and in pop culture --- most notably in The Simpsons.

The Simpsons cite religion quite a bit in their episodes --- many episodes have religion as the sole topic matter, such as "Homer the Heretic" and "Simpson's Bible Stories". Most of these episodes poke fun at religion, especially the Catholic Church.

"Homer the Heretic" is about how Homer decides to skip church on a freezing Sunday morning and has one of the best days of his life while committing nearly all of the deadly sins. After this bout of blasphemy, he is confronted by God in his dreams about his actions. The only explanation Homer has is, "What does it matter if I go to church if I worship you in my own way?" God replies, "You've got a point there", and Homer goes away scot-free. This essentially is a giant insult thrown at the Catholic Church, who dictates how you should worship God. Homer and God make the point that it really doesn't matter how you worship a God --- the Church is simply a middleman in this situation. There is no where within the Bible that says the Church must exist, and this episode shows how deceitful and unnecessary the Church is.

"Simpson's Bible Stories" has several different parts. The first, and most prominent, is that of Adam and Eve, who are portrayed by Homer and Marge. This episode is about the absurdity and inconsistencies within the Bible. The viewer can clearly tell that it's the same exact story --- however, the creators just added lines like when Adam is talking to Eve, "You're pretty up tight for a naked chick", and the "porno" and "bacon" trees. There are inconsistencies, such as how Adam eats a forbidden fruit without consequence, but when Eve eats it, there is immediate retaliation from a Flanders-based God.

Many accuse The Simpsons of being anti-religion, when this simply is not the case. They do not condemn religion --- rather, they point out ways in which the current religion structure can be improved upon. This mistrust in the current system is postmodern. They appreciate the idea of religion, but do not like the way the current system is set up. By pointing out the downfalls and irregularities, they are hoping for a better church in the future.



%%%%Works cited
\begin{workscited}

\bibent
Big Think. \textquotedblleft Neil deGrasse Tyson: Atheist or Agnostic?\textquotedblright Online video clip. \textit{YouTube}, 25 Apr. 2012. Web. 6 Oct. 2013. <http://www.youtube.com/watch?v=CzSMC5rWvos>

\bibent
Funk, Cary, and Gregory A. Smith. \textquotedblleft "Nones" on the Rise.\textquotedblright  \texit{Religion and Public Life}. Pew Research Center, 9 Oct. 2012. Web. 06 Oct. 2013. <http://www.pewforum.org/2012/10/09/nones-on-the-rise/>.

\end{workscited}
\end{flushleft}
\end{document}
\}